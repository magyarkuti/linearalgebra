% lualatex zsiga.tex
\documentclass[9pt, a4paper, showtrims, article]{memoir}
\usepackage{polyglossia}\setdefaultlanguage{english}
\usepackage[a4paper]{geometry}
\usepackage{amssymb, amsmath, amsthm}
\swapnumbers

\theoremstyle{plain}
\newtheorem{proposition}{Proposition}[chapter]
\newcommand{\parencite}{\cite}

\newcommand{\ip}[2]{\langle#1,#2\rangle}
\DeclareMathOperator{\rank}{rank}
\DeclareMathOperator{\crank}{crank}
\DeclareMathOperator{\rrank}{rrank}
\DeclareMathOperator{\srank}{srank}
\DeclareMathOperator{\lin}{lin}
\synctex=1
\begin{document}
\begin{proposition}
    Assume that a vector space has a finite spanning set having exactly $n$ vectors.
    Then every system of vectors including more than $n$ vector is linearly dependent.
\end{proposition}
    Let $\left\{ y_1,\ldots,y_n \right\}$ be a spanning set
    and assume that the system  $\left\{ x_1,\ldots,x_m \right\}$ has more vector than the spanning set has.
    Thus $m>n$.
%    It is enough to prove that the $m$ element system is dependent.
    By the definition of the spanning set: for all $1\leq k\leq m$ there exist coefficients 
    $\alpha_{j,k}, 1\leq j\leq n$, such that
    \[
        x_k=\sum_{j=1}^n\alpha_{j,k}y_j.
    \]
    Write a linear combination of the system $\left\{ x_1,\dots,x_m \right\}$ using the multipliers $\xi_1,\ldots,\xi_m$ specified later.
    \begin{eqnarray}
        \sum_{k=1}^m\xi_kx_k=
        \sum_{k=1}^m\sum_{j=1}^n\xi_k\alpha_{j,k}y_j=
        \sum_{j=1}^n\left( \sum_{k=1}^m\alpha_{j,k}\xi_k \right)y_j
        \label{eq:sys}
    \end{eqnarray}
    Consider the homogeneous system of linear equations belonging to the coefficients $\left( \alpha_{j,k} \right)$.
    Here $j=1,\ldots,n$ and $k=1,\ldots,m$.
    The assumption $m>n$ indicates the number of unknowns is greater than the number of equations of this linear system.
    Remember: this fact implicates the existence of a nontrivial solution of this system.
    Thus there exist numbers $\xi_1,\ldots,\xi_m$, for which not all of them is zero
    and at the same time for every $j=1,\ldots,n$ the equations
    \[
        \sum_{k=1}^m\alpha_{j,k}\xi_k=0
    \]
    hold.
    Thus if we use the above found unknowns ($\xi_k$) for the multipliers of (\ref{eq:sys}), then the right hand side of (\ref{eq:sys}) becomes zero.
    Consider now the left hand side of (\ref{eq:sys}). 
    It is a non trivial linear combination of the system $\left\{ x_1,\ldots,x_m \right\}$.
    It was to be proved.
\end{document}
