\documentclass[aspectratio=169,notheorems]{corvinusmetropolis}

\newtheorem{theorem}{Theorem}
%\newtheorem{proposition}[theorem]{Proposition}
%\newtheorem{lemma}[theorem]{Lemma}
\theoremstyle{definition}
\newtheorem{definition}[theorem]{Definition}
%\newtheorem{corollary}[theorem]{Corollary}
\newtheorem{example}[theorem]{Example}

\title{Quadratic forms}
\subtitle{The Gauss--Jordan elimination completes the square}
\author[Magyarkuti]{Gyula~Magyarkuti}
\email{magyarkuti@uni-corvinus.hu}

\institute[Corvinus University]
  {
  Department of Mathematics\\
  Corvinus University of Budapest\\
  Email: \putemail\\
  Download from:
  \url{http://web.uni-corvinus.hu/magyarkuti/\jobname .pdf}
  }
\date[Linear Algebra Fall week 6]{\today}
\titlegraphic{\hfill{\includegraphics[width=.2\textwidth]{corvinus_Department_of_Mathematics_logo_cmyk.eps}}}
\subject{Linear Algebra}
\keywords{Gauss--Jordan elimination, completing the square}

\newcommand{\R}{\mathbb{R}}
\newcommand{\lin}{\operatorname{lin}}
\begin{document}
\openingframes{Outline}
\section{Definition of quadratic forms}
\frame{
A quadratic form is a special multi-variable function.
\begin{definition}
    Let $A\in\mathbb{R}^{n\times n}$ be a symmetric matrix.
    Assume that $a_{k,j}$ denotes the appropriate element of that matrix.
    The function $Q:\mathbb{R}^n\to\mathbb{R}$
    \[
    Q\left( x_{1},x_2,\cdots,x_n \right)=
    \sum_{k=1}^n\sum_{j=1}^na_{k,j}x_kx_j
    \]
    is called $n$ variables quadratic function.
\end{definition}
If $n=1$ then $Q(x_1)=ax_1^2$.\\
If $n=2$ then
$$
Q\left( x_1,x_2 \right)=a_{1,1}x_1^2+a_{1,2}x_1x_2+a_{2,1}x_2x_1+a_{2,2}x_2^2.
$$
Compute the case when $n=3$.
}
\begin{frame}
    \begin{theorem}
        If $A\in\mathbb{R}^{n\times n}$ is a symmetric matrix then the quadratic form determined by 
        $A$ is
        \[
            Q(x)=x\cdot Ax
        \]
        for every $x\in\mathbb{R}^n.$
    \end{theorem}
    \begin{example}
        Write the symmetric matrix of quadratic form given before.
        Write the quadratic form of a symmetric matrix given before.
    \end{example}
    \begin{theorem}
        The quadratic form determined by $A+B$ is the same as the sum of quadratic forms determined by $A$ and $B$ respectively.
        Similarly,
        the quadratic form determined by $\alpha A$ is the same as the form determined by $A$ multiplied by $\alpha$.
    \end{theorem}
\end{frame}
\section{Dyadic decomposition}
\subsection{Dyad and completing the square}
\begin{frame}
    The dyad is a special symmetric matrix representing the perfect squares.
    \begin{definition}
        A symmetric matrix $A\in\mathbb{R}^{n\times n}$ is called \alert{\emph{dyad}}, 
        if there exists a vector
        $a\in\mathbb{R}^{n\times 1}$ for which
%        \metroset{block=fill}
%        \begin{block}{}
        \(
            A=a\cdot a^T
        \)
%        \end{block}
%        \metroset{block=transparent}
    \end{definition}
    \begin{theorem}
        A quadratic form is a complete square if and only if it is determined by a dyad.
        If $A=a\cdot a^T$ where $a=\left( a_1,a_2,\cdots,a_n \right)$, then
%        \metroset{block=fill}
%        \begin{block}{}
          \(
             Q\left( x_1,x_2,\cdots,x_n \right)=
              \left( a_1x_1+a_2x_2+\cdots+a_nx_n \right)^2,
          \)
%        \end{block}
%        \metroset{block=transparent}
        here $Q$ is the quadratic form determined by the dyad $A$.
    \end{theorem}
\end{frame}
\subsection{High school method for a few variable only}
\begin{frame}
    \frametitle{Decomposition}
    \metroset{block=fill}
    \begin{theorem}
        Every symmetric matrix is a linear combination of dyads. Thus
        every quadratic form is a linear combination of complete squares.
    \end{theorem}
    \metroset{block=transparent}
\begin{example}
    Completing the square, rewrite the following quadratic forms as a linear combination of perfect squares.\\
    $Q\left( x_1,x_2 \right)=x_1^2+4x_1x_2-5x_2^2,$\\
    $Q\left( x_1,x_2 \right)=4x_1^2+6x_1x_2+9x_2^2,$\\
    $Q\left( x_1,x_2 \right)=5x_1^2-6x_1x_2+x_2^2,$\\
    $Q\left( x_1,x_2,x_3 \right)=5x_1^2-6x_1x_2+x_1x_3,$\\
\end{example}
\end{frame}
\subsection{Professional method}
\begin{frame}
    \frametitle{Dyadic decomposition using Gauss--Jordan elimination}
    Every $n$-variable quadratic form can be decreased by a multiply of a dyad such that
    the remainder quadratic form has one variable less:
    \metroset{block=fill}
    \begin{theorem}
        Consider a symmetric matrix $A\in\mathbb{R}^{n\times n}$.
        Assume, that $a_{1,1}\neq 0$ and denote $d=\frac{1}{a_{1,1}}a_1$,
        where $a_1$ is the first column of matrix $A$.
        Then 
        \[
            R=A-a_{1,1}d\cdot d^T
        \]
        is a symmetric matrix, where the first row (and column) is the zero vector.
    \end{theorem}
    \metroset{block=transparent}
    \begin{proof}
        The sum of symmetric matrices is a symmetric matrix.
        If $\delta_k$ denotes the $k$-th coordinate of $d$, then
        \(
            r_{k,j}=
            a_{k,j}-a_{1,1}\delta_{k}\delta_j.
        \)
        If $k=1$ we obtain $r_{1,j}=a_{1,j}-a_{1,1}\cdot 1\cdot\frac{a_{1,j}}{a_{1,1}}=0$.
    \end{proof}
\end{frame}
\begin{frame}
    The $k,j$ term of the matrix of the remainder quadratic form is
    \metroset{block=fill}
    \begin{block}{}
    \[
        r_{k,j}=
        a_{k,j}-a_{1,1}\delta_k\delta_j=
        a_{k,j}-a_{1,1}\frac{a_{k,1}}{a_{1,1}}\delta_j=
        a_{k,j}-a_{k,1}\delta_j.
    \]
    \end{block}
    If $k>1$ and $j>1$, then the above formula just computes
    the new coordinates of the $j$-th column after the first column has entered to the base at the first position.
    Thus we proved the following theorem.
    \begin{theorem}
        Let $A\in\mathbb{R}^{n\times n}$ be a symmetric matrix and assume that $a_{1,1}\neq 0$.
        After separating the first complete square, as in the theorem above, the matrix of the remainder quadratic form
        is the bottom right $n-1\times n-1$ matrix of the transformation table, when $a_{1,1}$ is the pivot term.
    \end{theorem}
\end{frame}
\section{Problems}
\begin{frame}
    \frametitle{Problems}
    \begin{block}{ Write the following quadratic forms as a linear combination of complete squares.}
        \begin{enumerate}
            \item 
            $Q\left( x_2,x_3 \right)=13x_2^2-4x_2x_3+8x_3^2$,
            \item
            $Q\left( x_1,x_2,x_3 \right)=5x_1^2+6x_2^2+4x_3^2-4x_1x_2-4x_1x_3$,
            \item
            $Q\left( z_1,z_2,z_3 \right)=2z_1^2+\frac{3}{2}z_3^2+2z_1z_2-4z_1z_3+2z_2z_3$,
            \item
            $Q\left( x_1,x_2,x_3,x_4 \right)=2x_1^2+2x_2^2+2x_3^2+2x_4^2-4x_1x_2+2x_1x_4+2x_2x_3-4x_3x_4$
        \end{enumerate}
    \end{block}
\end{frame}
\section{Definiteness of a quadratic form}
\begin{frame}
    \frametitle{Definiteness}
    Classify a quadratic form with respect to to the range!
    \begin{definition}
        Let $Q:\mathbb{R}^n\to\mathbb{R}$ be a non-zero quadratic form. It is called
        \begin{description}
            \item[positive definite,] if $Q\left( x \right)>0$ for every $x\in\mathbb{R}^n,x\neq 0$;
            \item[positive semidefinite,] if $ Q\left( x \right)\geq 0$ for every $x\in\mathbb{R}^n$,
                but there exists a $z\in\mathbb{R}^n, z\neq 0$ such that $Q\left( z \right)=0$;
            \item[negative definite,] if $ Q\left( x \right)<0$ for every $x\in\mathbb{R}^n,x\neq 0;$
            \item[negative semidefinite,] if $ Q\left( x \right)\leq 0$ for every $x\in\mathbb{R}^n$,
                but there exists a $z\in\mathbb{R}^n, z\neq 0$ such that $Q\left( z \right)=0$;
            \item[indefinite,] if there exist two vectors $x,y\in\mathbb{R}^n$ for which
                $Q\left( x \right)>0$ but $Q\left( y \right)<0$.
        \end{description}
    \end{definition}
\end{frame}
\subsection{Summary}
\begin{frame}
    \frametitle{Definiteness after completing the square}
    It is not hard to decide the definiteness of a quadratic form if it is given with its dyad decomposition.
    Let $Q$ be an $n$-variable non-zero quadratic form.\\
    If the dyad decomposition 
    \begin{itemize}
        \item includes positive and negative pivot terms then $Q$ is an \alert{\emph{indefinite}} quadratic form.
    \end{itemize}
    Now assume the dyad decomposition includes exactly $n$ complete squares.
    If
    \begin{itemize}
        \item all pivot terms are positive numbers then $Q$ is \alert{\emph{positive definite}};
        \item all pivot terms are negative numbers then $Q$ is \alert{\emph{negative definite}}.
    \end{itemize}
\end{frame}
\begin{frame}
    \frametitle{Definiteness after completing the square \ldots}
    If the number of dyads is less than $n$, 
    the same characterization holds true but the quadratic form is semidefinite only:\\
    Thus if
    \begin{itemize}
        \item all pivot terms are positive numbers then $Q$ is \alert{\emph{positive semidefinite}};
        \item all pivot terms are negative numbers then $Q$ is \alert{\emph{negative semidefinite}}.
    \end{itemize}
\end{frame}
\closingframes{Thank you for your attention!}
\end{document}
