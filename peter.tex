\documentclass[9pt, a4paper, showtrims]{memoir}
\usepackage{polyglossia}\setdefaultlanguage{magyar}
\usepackage[a4paper]{geometry}
\usepackage{amssymb, amsmath, amsthm}
\swapnumbers

\theoremstyle{plain}
\newtheorem{proposition}{állítás}[chapter]
\theoremstyle{definition}
\newtheorem{definition}[proposition]{definíció}
\newtheorem{defprop}[proposition]{definíciós-állítás}
\newtheorem{corollary}[proposition]{következmény}
\newtheorem{note}[proposition]{megjegyzés}
\newcommand{\parencite}{\cite}

\newcommand{\ip}[2]{\langle#1,#2\rangle}
\DeclareMathOperator{\rank}{rank}
\DeclareMathOperator{\crank}{crank}
\DeclareMathOperator{\rrank}{rrank}
\DeclareMathOperator{\srank}{srank}
\DeclareMathOperator{\lin}{lin}
\synctex=1
\begin{document}
\chapter{Kvadratikus alakok definitségének egy naiv tárgyalása}
A kvadratikus alaknak mondunk egy speciális többváltozós függvényt.
\begin{definition}
    Legyen $A\in\mathbb{R}^{n\times n}$ egy szimmetrikus mátrix.
    Amint az szokásos jelölje $a_{k,j}$ a mátrix $i,j$ pozicíojának elemét.
    Definiálja a $Q_A:\mathbb{R}^n\to\mathbb{R}$ függvényt
    \[
    Q\left( x_{1},x_2,\cdots,x_n \right)=
    \sum_{k=1}^n\sum_{j=1}^na_{k,j}x_kx_j.
    \]
    Ekkor $Q_A$ az $A$ szimmetrikus mátrixhoz tartozó kvadratikus alak.
\end{definition}
Például $n=1$ és az $A$ mátrix egyetlem eleme az $a\in\mathbb{R}$ szám, akkor $Q(x_1)=ax_1^2$.
Ha $n=2$ mellett $A\in\mathbb{R}^{2\times 2}$, akkor
$$
Q\left( x_1,x_2 \right)=a_{1,1}x_1^2+a_{1,2}x_1x_2+a_{2,1}x_2x_1+a_{2,2}x_2^2.
$$
Gyakorlásképpen írjuk fel milyen három változós kvadratikus alakok lehetségesek.
\begin{proposition}
    Legyen $A\in\mathbb{R}^{n\times n}$ egy szimmetrikus mátrix.
    Ekkor az $A$ által meghatározott kvadratikus alakra
    \[
        Q(x)=\ip{x}{Ax}
    \]
    tetszőleges $x\in\mathbb{R}^n.$
\end{proposition}
Ebből az alakból már világos, hogy az $A+B$ összeghez tartozó kvadratikus alak azonos az $A$-hoz és a $B$-hez
tartozó kvadratikus alakok összegével, 
és hasonlóan az $\alpha A$ mátrixhoz tartozó kvadratikus alakot úgy is megkaphatjuk, hogy az $A$-hoz tartozó kvadratikus alakot mint $n$-változós függvényt szorozzuk az $\alpha$ számmal.
Formálisabban:
\begin{proposition}
    Legyenek $A,B\in\mathbb{R}^{n\times n}$ szimmetrikus mátrixok és $\alpha,\beta\in\mathbb{R}$.
    Ekkor
    \[
    Q_{\alpha A+\beta B}=\alpha Q_A+\beta Q_B.
    \qedhere
    \]
\end{proposition}
A nemzérus kvadratikus alakokat az értékkészletük alapján osztályozzuk.
\begin{definition}
    Legyen $Q:\mathbb{R}^n\to\mathbb{R}$ egy nemzérus kvadratikus alak.
    Azt mondjuk, hogy ez
    \begin{description}
        \item[{\normalfont \emph{pozitív definit,}}] ha $Q\left( x \right)>0$ minden $x\in\mathbb{R}^n,x\neq 0$;
        \item[{\normalfont\emph{pozitív szemidefinit,}}] ha $ Q\left( x \right)\geq 0$ minden $x\in\mathbb{R}^n$,
            de létezik egy $z\in\mathbb{R}^n, z\neq 0$ vektor, amelyre $Q\left( z \right)=0$;
        \item[{\normalfont\emph{negatív definit,}}] ha $ Q\left( x \right)<0$ minden $x\in\mathbb{R}^n,x\neq 0;$
        \item[{\normalfont\emph{negatív szemidefinit,}}] ha $ Q\left( x \right)\leq 0$ minden $x\in\mathbb{R}^n$,
            de létezik egy $z\in\mathbb{R}^n, z\neq 0$ vektor, amelyre $Q\left( z \right)=0$;
        \item[{\normalfont\emph{indefinit,}}] ha vannak olyan $x,y\in\mathbb{R}^n$ vektorok, amnelyekre
            $Q\left( x \right)>0$  és $Q\left( y \right)<0$.
    \end{description}
\end{definition}
\subsection{A szimmetrikus diádok szerepe}
Emlékezzünk, hogy $n\times n$ méretű szimmetrikus diádnak nevezünk egy $\mathbb{R}^n$ beli vektornak
oszlopalakú és soralakú mátrixszorzatát. 
Kicsit formálisabban az $A=a\cdot a^T$ tetszőleges $a\in\mathbb{R}^{n\times 1}$ mellett.
A diádok tehát speciális szimmetrikus mátrixok.
Látható, hogy egy szimmetrikus mátrix pontosan akkor diád, 
ha az 1 rangú szimmetrikus mátrix.

A szimmetrikus diádok azért érdekesek számunkra, 
mert éppen az ezekhez tartozó kvadratikus alakok a teljes négyzetek.
\begin{proposition}
   Egy kvadratikus alak pontosan akkor teljes négyzet, ha az őt reprezentáló szimetrikus mátrix egy diád.
   Pontosabban:
   Ha $A=a\cdot a^T$, ahol $a=\left( a_1,a_2,\ldots,a_n \right)$, akkor
      \(
         Q_A\left( x_1,x_2,\ldots,x_n \right)=
         \left( a_1x_1+a_2x_2+\ldots+a_nx_n \right)^2.
      \)
\end{proposition}

\section{Szimmetrikus mátrix diád felbontása}
Be fogjuk bizonyítani, hogy 
\begin{proposition}
    Minden nem zérus szimmetrikus mátrix előáll mint szimmetrikus diádok lineáris kombinációja.
    Sőt, ha egy szimmetrikus mátrix rangja $r$, 
    akkor az előáll mint $r$ db.~szimmetrikus diád lineáris kombinációja, 
    ahol a diádokat alkotó vektorok rendszere lineárisan független rendszert alkot.
\end{proposition}
A bizonyítás úgy folyik, hogy konkrét algoritmust adunk a felbontásra.

Az algoritmus első mozzanata annak észrevétele, 
hogy egy szimmetrikus mátrixból leválasztható egy szimmetrikus diád olyan módon,
hogy a maradék szimmetrikus mátrixhoz tartozó kvadratikus alak már egy változóval kevesebbett tartalmazzon:
\begin{proposition}
    Tekintsünk egy $A\in\mathbb{R}^{n\times n}$ szimmetrikus mátrixot.
    Tegyük fel, hogy a balfelső elemre $a_{1,1}\neq 0$.
    Jelölje $d=\frac{1}{a_{1,1}}a_1$, ahol $a_1$ a mátrix első oszlopa $A$.
    Ekkor
    \[
        R=A-a_{1,1}d\cdot d^T
   \]
   egy olyan szimmetrikus mátrix, 
   ahol az első oszlop és első sor kizárólag zérus elemeket tartalmaz.
\end{proposition}
\begin{proof}
    Az persze nyilvánvaló, hogy szimmetrikus mátrixok egy lináris kombinációja is szimmetrikus marad.
    Jelölje $\delta_k=\frac{a_{k,1}}{a_{1,1}}=\frac{a_{1,k}}{a_{1,1}}$ a $d$ vektor $k$-adik koordinátáját.
    Ekkor 
    \[
        r_{k,j}=
        a_{k,j}-a_{1,1}\delta_k\delta_j=
        a_{k,j}-a_{1,1}\frac{a_{k,1}}{a_{1,1}}\delta_j=
        a_{k,j}-a_{k,1}\delta_j.
        \tag{\dag}
    \]
    Ha $j=1$, akkor itt $r_{k,1}=a_{k,1}-a_{k,1}\delta_1=0$.
    Látjuk, hogy az $R$ mátrix első oszlopa csupa zérust tartalmaz. 
    No de, az $R$ egy szimmetrikus mátrix, ergo az első sorban is csak zérus elemek vannak.
    \end{proof}
    Illusztrációként nézzük a következő kvadratikus alakot:
    \[
        Q\left( z_1,z_2,z_3 \right)
        =
        2z_1^2+\frac{3}{2}z_3^2+2z_1z_2-4z_1z_3+2z_2z_3.
    \]
    Ezt a kvadratikus alakot az 
    \(A=
    \begin{pmatrix}
        2&1&-2\\
        1&0&1\\
        -2&1&\frac{3}{2}
    \end{pmatrix}
    \)
    szimmetrikus mátrix reprezentálja, 
    legyen tehát 
    \(d=
    \begin{bmatrix}
        1\\ \frac{1}{2}\\ -1
    \end{bmatrix}.
    \)
    Így 
    \[
        R=A-2d\cdot d^T
        =
        \begin{pmatrix}
        2&1&-2\\
        1&0&1\\
        -2&1&\frac{3}{2}
        \end{pmatrix}
        -2
        \begin{bmatrix}
        1\\ \frac{1}{2}\\ -1
        \end{bmatrix}
        \cdot
        \begin{bmatrix}
        1& \frac{1}{2}& -1
        \end{bmatrix}
        =
        \begin{pmatrix}
            0&0&0\\
            0&-\frac{1}{2}&2\\
            0&2&-\frac{1}{2}
        \end{pmatrix}
    \]
    No persze $A=2d\cdot d^T+R$.
    Áttérve a kvadratikus alakokra
    \(
        Q\left( z_1,z_2,z_3 \right)=2\left( z_1+\frac{1}{2}z_2-z_3 \right)^2+Q_R\left( z_1,z_2,z_3 \right),
    \)
    ahol $Q_R$ már nem függ a $z_1$ változótól, 
    hiszen $Q_R\left( z_1,z_2,z_3 \right)=-\frac{1}{2}z_2^2-\frac{1}{2}z_3^2+4z_2z_3$.
    \\
    Sikeresen megszabadultunk az első változótól, elegendő tehát a 
    \(
    \begin{pmatrix}
        -\frac{1}{2}&2\\
        2&-\frac{1}{2}
    \end{pmatrix}
    \)
    mátrix reprezentálta kétváltozós kvadratikus alakkal foglalkozni, amelynek változóit most $z_2$ és $z_3$ jelöli.
    Válasszunk le egy újabb szimmetrikus diádot:
    \[
        \begin{pmatrix}
            -\frac{1}{2}&2\\
            2&-\frac{1}{2}
        \end{pmatrix}
        -\left( -\frac{1}{2} \right)
        \begin{bmatrix}
            1\\-4
        \end{bmatrix}
        \cdot
        \begin{bmatrix}
            1&-4
        \end{bmatrix}
        =
        \begin{pmatrix}
            0&0\\
            0&\frac{15}{2}
        \end{pmatrix}.
    \]
    Lefordítva ezt a kvadratikus alakok nyelvére azt kaptuk,
    hogy
    \(
        Q_R\left( z_1,z_2,z_3 \right)
        =
        -\frac{1}{2}\left( z_2-4z_3 \right)^2+\frac{15}{2}z_3^2.
    \)
    Összefoglalva tehát
    \[
        Q\left( z_1,z_2,z_3 \right)
        =
        2\left( z_1+\frac{1}{2}z_2-z_3 \right)^2
        -\frac{1}{2}\left( z_2-4z_3 \right)^2
        +\frac{15}{2}z_3^2.
    \]
    Az $A$ mátrix diádfelbontását is leolvashatjuk a fenti számolásból:
    \[
        A
        =
        2
        \begin{bmatrix}
        1\\ \frac{1}{2}\\ -1
        \end{bmatrix}
        \cdot
        \begin{bmatrix}
        1& \frac{1}{2}& -1
        \end{bmatrix}
        -\frac{1}{2}
        \begin{bmatrix}
            0\\1\\-4
        \end{bmatrix}
        \cdot
        \begin{bmatrix}
            0&1&-4
        \end{bmatrix}
        +
        \frac{15}{2}
        \begin{bmatrix}
            0\\0\\1
        \end{bmatrix}
        \cdot
        \begin{bmatrix}
            0&0&1
        \end{bmatrix}.
    \]

Voltaképpen készen is lennénk, mert
ez a módszer minden olyan esetben megadja a szimetrikus diádokra való felbontást,
mikor az algoritmus kezdő és minden további lépésében a maradék kvadratikus alak mátrixának balfelső eleme nem zérus.%
\footnote{
Evvel a problémával most ne foglalkozzunk később ki fog derülni, 
hogy könnyen megszabadulhatunk e kellemetlennek tűnő feltéteteltől.}

A maradék kvadratikus alak mátrixa, az $R=A-a_{1,1}d\cdot d^T$, a fentinél sokkal
hatékonyabban számolható.
Nézzük rá az állítás igazolásában kiemelt $(\dag)$ azonosságra de úgy,
hogy gondoljunk közben a Gauss\,--\,Jordan-eliminációra:
    \[
        r_{k,j}=
        a_{k,j}-a_{k,1}\delta_j.
        \tag{\dag}
    \]
Tudjuk, hogy az $R$ maradék első oszlopa és első sora zérus.
Ezért a kérdés csak az, 
hogy e mátrix jobb alsó $n-1\times n-1$ méretű részének mik az elemei.
A fenti formulát kell ehhez alkalmazni, mikor $k>1$ és $j>1$.
Ismerjük fel, hogy (\dag)  éppen az eliminációs algoritmus számolási szabálya abban az esetben, 
amikor a balfelső elem a pivot elem és a pivot elemet nem tartalmazó sorok elemeit számoljuk.
Ez azt jelenti, hogy az $A-a_{1,1}d\cdot d^T$ művelet diád szorzás és mátrix kivonás helyett Gauss\,--\,Jordan-eliminációval is számolható.
Bebizonyítottuk hát a következő állítást.
\begin{proposition}
   Legyen $A\in\mathbb{R}^{n\times n}$ egy olyan szimmetrikus mátrix,
   amelyre $a_{1,1}\neq 0$.
   Legyen $R$ az első diád leválasztása után maradó kvadratikus alak mátrixa,
   azaz 
   $$R=A-a_{1,1}d\cdot d^T,$$
   ahol $d$ az $A$ mátrix első oszlopának az $a_{1,1}$ reciprokával vett szorzata.
   Tudjuk, hogy $R$ első oszlopa és sora csak zérus elemeket tartalmaz, viszont e
   mátrix a jobb alsó $\left(n-1\times n-1  \right)$ méretű része megkapható
   az $A$ mátrixon végzett Gauss\,--\,Jordan-eliminációval, 
   olyan módon hogy a balfelső $a_{1,1}$ elemet választjuk pivot elemnek, 
   és az elimináció első sorát eldobjuk.
\end{proposition}
Ennek fényében az előző példa számolása radikálisan egyszerűsíthető. Nézzük újra.
\[
\begin{array}{c|ccc}
     & z_1       & z_2 & z_3 \\
     \hline
     & \boxed{2} & 1   & -2  \\
     & 1         & 0   & 1   \\
     & -2        & 1   & 3/2 \\
    \hline
    & \delta    & \frac{1}{2}   & -1
\end{array}
\implies
\begin{array}{c|ccc}
     & z_1       & z_2  & z_3 \\
     \hline
     & &                &       \\
     & & \boxed{-1/2}   & 2     \\
     & &            2   & -1/2  \\
     \hline
     & & \delta         & -4
\end{array}
\implies
\begin{array}{c|ccc}
     & z_1       & z_2 & z_3 \\
     \hline
     & &   &     \\
     & &   &     \\
     & &   & \boxed{15/2}    \\
    \hline
    \phantom{\delta}
\end{array}
\]
Ezután csak értelmeznünk kell a fenti eliminációt.
A második táblázat az elő diád leválasztása után maradt kvadratikus alak mátrixa.
A leválasztott diád az első táblázat segédsorából olvasható le, 
és a diád szorzója az első táblázatban választott pivot elem.
Tehát
\[
    \begin{pmatrix}
        2&1&-2\\
        1&0&1\\
        -2&1&\frac{3}{2}
    \end{pmatrix}
    =
    2
    \begin{bmatrix}
        1\\1/2\\-1
    \end{bmatrix}
    \cdot
    \begin{bmatrix}
        1&1/2&-1
    \end{bmatrix}
    +
    \begin{pmatrix}
        0&0&0\\
        0&-1/2&2\\
        0&2&-1/2
    \end{pmatrix}.
\]
A harmadik táblázat az előző maradékból leválasztott kvadratikus alak mátrixa.
A leválasztott diád a második táblázat segédsorából olvasható le, 
és a szorzó a második táblázat pivot eleme.
Tehát
\[
    \begin{pmatrix}
        0&0&0\\
        0&-1/2&2\\
        0&2&-1/2
    \end{pmatrix}
    =
    -\frac{1}{2}
    \begin{bmatrix}
        0\\1\\-4
    \end{bmatrix}
    \cdot
    \begin{bmatrix}
        0&1&-4
    \end{bmatrix}
    +
    \begin{pmatrix}
        0&0&0\\
        0&0&0\\
        0&0&15/2
    \end{pmatrix}
\]
Világos, hogy az utolsó maradék $15/2$ szerese a 
\(
\begin{pmatrix}
    0\\0\\1
\end{pmatrix}
\)
vektor által meghatározott diádnak, tehát a diád felbontás a fenti eliminációs táblázatból azonnal leolvasható:
\[
    A
    =
    2
    \begin{bmatrix}
    1\\ \frac{1}{2}\\ -1
    \end{bmatrix}
    \cdot
    \begin{bmatrix}
    1& \frac{1}{2}& -1
    \end{bmatrix}
    -\frac{1}{2}
    \begin{bmatrix}
        0\\1\\-4
    \end{bmatrix}
    \cdot
    \begin{bmatrix}
        0&1&-4
    \end{bmatrix}
    +
    \frac{15}{2}
    \begin{bmatrix}
        0\\0\\1
    \end{bmatrix}
    \cdot
    \begin{bmatrix}
        0&0&1
    \end{bmatrix}
\]
A diád felbontás a kvadratikus alakra nézve azt jelenti, 
hogy előállítottuk azt olyan teljes négyzetek lineáris kombinációjaként,
ahol a teljes négyzeteknek megfelelő diádokat generáló vektorok rendszere linárisan független rendszert ad.
Konkrétan itt
\[
    2z_1^2+\frac{3}{2}z_3^2+2z_1z_2-4z_1z_3+2z_2z_3
    =
    2\left( z_1+\frac{1}{2}z_2-z_3 \right)^2
    -\frac{1}{2}\left( z_2-4z_3 \right)^2
    +\frac{15}{2}z_3^2.
\]

\paragraph{Gyakorlat.}
Írjuk fel a 
\(
Q\left( x_1,x_2,x_3 \right)=
x_1^2+3x_2^2+6x_3^2+
2x_1x_2+2x_1x_3+
6x_2x_3
\)
kvadratikus alakot teljesnégyzetek lineáris kombinációjaként.
Az elimináció lépései:
\[
\begin{array}{c|ccc}
     & x_1       & x_2 & x_3 \\
     \hline
     & \boxed{1} & 1   & 1   \\
     & 1         & 3   & 3   \\
     & 1         & 3   & 6   \\
    \hline
    & \delta     & 1   & 1
\end{array}
\implies
\begin{array}{c|ccc}
     & x_1& x_2  & x_3 \\
     \hline
     & &                &     \\
     & & \boxed{2}      & 2   \\
     & & 2              & 5   \\
     \hline
     & & \delta         & 1
\end{array}
\implies
\begin{array}{c|ccc}
     & x_1 & x_2 & x_3      \\
     \hline
     &     &     &          \\
     &     &     &          \\
     &     &     & \boxed{3}\\
    \hline
    \phantom{\delta}
\end{array}
\]
Ez azt jelenti, hogy 
\(
Q\left( x_1,x_2,x_3 \right)
=
\left( x_1+x_2+x_3 \right)^2
+2\left( x_2+x_3 \right)^2
+3x_3^2.
\)
Az is világos, hogy a három pivot elem pozitivitása szerint a kvadratikus alak pozitív definit,
hiszen az
\begin{eqnarray*}
    x_1+x_2+x_3&=& 0\\
    x_2+x_3&=& 0\\
    x_3&=& 0
\end{eqnarray*}
homogén lineáris egyenletrendszernek, 
az együttható mátrix oszlopainak lináris függetlensége miatt,
csak triviális megoldása van!

\subsubsection{Zérus elem a bal felső sarokban}
Hangsúlyoznunk kell, hogy az algoritmusban a pivot elem megválasztása nem opcionális. 
A soron következő diád leválasztásához mindig a balfelső elemet kell választanunk.
Persze kérdés mi történik, ha a balfelső elem zérus?
Ennek illusztrációjaként nézzük az alábbi kvadratikus alakot.
\[
    Q\left( x_1,x_2,x_3 \right)=
    3x_2^2+
    2x_1x_2+2x_1x_3+
    6x_2x_3
\]
Persze el sem tudjuk kezdeni az algoritmust, hiszen a balfelső elem zérus.
Azt vegyük észre, hogy át tudjuk címkézni a változókat úgy, 
hogy a kapott kvadratikus alakban az első változónak legyen négyzete.
Ehhez csak az $x_1$ és az $x_2$ változókat kell felcserélnünk.
Vezessük be tehát a $z_1=x_2$, $z_2=x_1$ és $z_3=x_3$ jelöléseket.
Ezek szerint 
\[
    3x_2^2+
    2x_1x_2+2x_1x_3+
    6x_2x_3
    =
    3z_1^2+
    2z_2z_1+2z_2z_3+
    6z_1z_3
\]
Így viszont már indulhat is az elimináció:
\[
\begin{array}{c|ccc}
     & z_1       & z_2 & z_3 \\
     \hline
     & \boxed{3} & 1   & 3   \\
     & 1         & 0   & 1   \\
     & 3         & 1   & 0   \\
    \hline
    & \delta     & 1/3   & 1 
\end{array}
\implies
\begin{array}{c|ccc}
     & z_1& z_2  & z_3 \\
     \hline
     & &                &     \\
     & & \boxed{-1/3}      & 0   \\
     & & 0              & -3   \\
     \hline
     & & \delta         & 0
\end{array}
\implies
\begin{array}{c|ccc}
     & z_1 & z_2 & z_3      \\
     \hline
     &     &     &          \\
     &     &     &          \\
     &     &     & \boxed{-3}\\
    \hline
    \phantom{\delta}
\end{array}
\]
A teljes négyzetek lineáris kombinációját leolvasva:
\[
    3z_1^2+
    2z_2z_1+2z_2z_3+
    6z_1z_3
    =
    3\left( z_1+\frac{1}{3}z_2+z_3 \right)^2-\frac{1}{3}z_2^2-3z_3^2
    =
    3\left( \frac{1}{3}x_1+x_2+x_3 \right)^2-\frac{1}{3}x_1^2-3x_3^2,
\]
ami a diádfelbontásra nézve azt jelenti, hogy
\[
    \begin{pmatrix}
        0&1&1\\
        1&3&3\\
        1&3&0
    \end{pmatrix}
    =
    3
    \begin{bmatrix}
        1/3\\1\\1
    \end{bmatrix}
    \cdot
    \begin{bmatrix}
        1/3&1&1
    \end{bmatrix}
    -\frac{1}{3}
    \begin{bmatrix}
        1\\0\\0
    \end{bmatrix}
    \cdot
    \begin{bmatrix}
        1&0&0
    \end{bmatrix}
    -3
    \begin{bmatrix}
        0\\0\\1
    \end{bmatrix}
    \cdot
    \begin{bmatrix}
        0&0&1
    \end{bmatrix}.
\]
Ha ezt az átcímkézési trükköt megértettük, akkor még egyszerűbb jelöléseket is alkalmazhatunk.
Végül is csak az volt a lényeg, hogy az első két változót felcseréltük. 
Jelezhetjük ezt a táblázatban a változók indexével is. 
Ugyanez mégegyszer:
\[
\begin{array}{c|ccc}
     & x_1       & x_2 & x_3 \\
     \hline
     \phantom{\boxed{3}} & 0 & 1   & 1   \\
     & 1 & 3   & 3   \\
     & 0 & 3   & 0   \\
    \hline
    & \phantom{\delta}     &\phantom{1/3}    &\phantom{1}
\end{array}
\implies
\begin{array}{c|ccc}
     & x_2       & x_1 & x_3 \\
     \hline
     & \boxed{3} & 1   & 3   \\
     & 1         & 0   & 1   \\
     & 3         & 1   & 0   \\
    \hline
    & \delta     & 1/3   & 1 
\end{array}
\implies
\begin{array}{c|ccc}
     & x_2& x_1  & x_3 \\
     \hline
     & &                &     \\
     & & \boxed{-1/3}      & 0   \\
     & & 0              & -3   \\
     \hline
     & & \delta         & 0
\end{array}
\implies
\begin{array}{c|ccc}
     & x_2 & x_1 & x_3      \\
     \hline
     &     &     &          \\
     &     &     &          \\
     &     &     & \boxed{-3}\\
    \hline
    \phantom{\delta}
\end{array}
\]
A változók indexére ügyelve kell leolvasnunk a teljesnégyzeteket, és a legegszerűbb, ha a teljesnégyzetekből olvassuk le a diád felbontást.

Már csak azt kell meggondolnunk, hogyan kaptuk az első táblázatból a másodikat?
Azt kell látnunk, hogy az $i$-edik és a $j$-edik változó felcserélése avval jár, 
hogy a mátrixban felcseréljük az $i$-edik sort a $j$-edik sorral és az $i$-edik oszlopot a $j$-edik oszloppal.
Persze, hiszen a mátrixban minden az $i,k$ és a $j,k$ pozícióban álló szám helyet cserél, 
és hasonlóan minden a $k,i$ és a $k,j$ pozícióban álló szám is helyet cserél minden $k$ mellett.
A többi elem viszont nem változik.
Az első rész az $i$ és $j$ indexű sorok felcserélését jelenti, míg a második rész az $i$ és $j$ indexű oszlopok kicseréléséből áll.
A többi sort és oszlopot nem éri változás.

Látjuk tehát, hogy ha a balfelső elem zérus, de a diagonálisnak van nem zérus eleme, 
akkor a most tanult felcseréléssel a nem zérus elem a balfelső pozícióba hozható, így az algoritmusunk továbbra is működik.

\subsubsection{A diagonálisban minden elem zérus}
Most nézzük azt az esetet, amikor a diagonális minden eleme zérus, de van nem zéró elem.

Először is azt vegyük észre, hogy ha a feladat a kvadratikus alak definitségének meghatározása, 
akkor készen is vagyunk mert indefinit kvadratikus alakkal állunk szemben.
Ha ugyanis a $k,l$ pozíció tartalmaz nem zérus elemet, 
akkor a kvadratikus alak összes többi változóját állítsuk zérusra és a $k$-adik és az $l$-edik változót vizsgáljuk.
Így a kvadratikus alakunk az $2a_{k,l}x_kx_l$ kifejezésre egyszerűsödik, 
amely persze indefinit, hiszen az értékkészlete negatív és pozitív elemet egyaránt tartalmaz.

A teljesség kedvéért megmutatjuk, 
hogy az algoritmusunk ebben az esetben is használható.
Ha az $a_{k,l}\neq 0$, akkor vezessük be a 
\[
    z_l=x_l+x_k\qquad z_j=x_j, \text{ ha }j\neq l
\]
változókat.
Ekkor $k\neq l$-et is figyelembe véve
\begin{multline*}
z_kz_l=x_k\left( x_l+x_k \right)=x_kx_l+x_k^2=z_k^2+x_kx_l,
\text{ majd $j\neq k$, $j\neq l$ mellett: }\\
z_jz_l=x_j\left( x_l+x_k \right)=x_jx_l+x_jx_k=x_jx_l+z_jz_k.
\end{multline*}
Ez azt jelenti, hogy a $x_kx_l$ kifejezést a $(z_kz_l-z_k^2)$-re kell cserélnünk,
és minden további $x_jx_l$ alakú kifejezést a $(z_jz_l-z_jz_k)$ kifejezéssel kell kicserélnünk.
A többi változó indexe nem változik.
Így pontsan egy négyzetet kapunk, 
tehát a mátrix diagonálisában a $k$-adik sorban jelenik meg egy nem zérus elem.
Ahogyan azt az előzőekben már megértettük ezt az elemet kell a balfelső pozícióba csempészni, 
és az algoritmusunk már működik is.

\section{Definitség osztályozása a teljes négyzetek segítségével}
Miután elkészült a kvadratikus alakot reprezentáló mátrix diádfelbontása,
a kvadratikus alak definitsége az alábbiak szerint osztályozható.
Legyen tehát $Q$ egy $n$ változós nemzérus kvadratikus alak.\\
Ha a diád felbontás
\begin{itemize}
    \item tartalmaz pozitív és negatív pivot elemet, 
        akkor $Q$ \emph{indefinit}.
\end{itemize}
Most tegyük fel, hogy $Q$ felbontása pontosan $n$ teljes négyzetet tartalmaz.
Ha
\begin{itemize}
    \item az összes pivot elem pozitív,
        akkor $Q$ \emph{pozitív definit};
    \item az összes pivot elem negatív,
        akkor $Q$ \emph{negatív definit};
\end{itemize}
Ha $Q$ felbontása $n$-nél kevesebb teljes négyzetből áll,
akkor nagyon hasonló karakterizációt kapunk, de a mátrix szingulátis, ezért csak szemidefinitség jöhet szóba.
Tehát ha
\begin{itemize}
\item az összes pivot elem pozitív,
    akkor $Q$ \emph{pozitív szemidefinit};
\item az összes pivot elem negatív,
    akkor $Q$ \emph{negatív szemidefinit};
\end{itemize}

\chapter{Rang-tétel mégegszer}

\begin{definition}[rang, oszloprang, sorrang, feszítőrang]\index{vektorrendszer rangja}\index{oszloprang}\index{sorrang}\index{feszítőrang}
    Egy véges vektorrendszer \emph{rangján} a vektorrendszer generálta altér dimenzióját értjük.
    Egy mátrix \emph{oszloprangján} a mátrix oszlopai mint vektorrendszer rangját értjük.
    Egy mátrix \emph{sorrangján} a mátrix sorai mint vektorrendszer rangját értjük.
    Ha $A\in\mathbb{F}^{n\times m}$ nemzérus mátrix,
    akkor legkisebb olyan $r$ számot, amelyre
    létezik $B\in\mathbb{F}^{n\times r}$ és $C\in\mathbb{F}^{r\times m}$ mátrix úgy, hogy 
    $A=BC$ szorzatfelbontás teljesül,
    az $A$ mátrix \emph{feszítőrangjának} nevezzük.

    Jelölések: Ha $\left\{ x_1,\ldots,x_m \right\}$ a szóban forgó vektorrendszer, akkor
    \[
        \rank\left\{ x_1,\ldots,x_m \right\}=\dim\lin\left\{ x_1,\ldots,x_m \right\}
    \]
    Ha $A\in\mathbb{F}^{n\times m}$ egy mátrix,
    akkor 
    \[
        \crank{A}=\rank\left\{ [A]^{j}:j=1,\ldots,m \right\},
        \quad
        \rrank{A}=\rank\left\{ [A]_k:k=1,\ldots,n\right\},
    \]
    továbbá $\srank{A}$ jelöli a feszítőrangját $A$-nak.
\end{definition}
\begin{proposition}(Rang-tétel)\label{pr:rang}
    Tetszőleges test feletti tetszőleges mátrix mellett, a fent bevezetett három rang-koncepció azonos.

    Formálisabban: Minden $A\in\mathbb{F}^{n\times m}$ mellett
    \[
        \crank{A}=\srank{A}=\rrank{A}\qedhere
    \]
\end{proposition}
\begin{proof}
    Induljunk ki a feszítőrang fogalmából.
    Legyen $r=\srank{A}$, és 
    \[
        A=BC,\tag{\dag}
    \]
    ahol $B\in\mathbb{F}^{n\times r},C\in\mathbb{F}^{r\times m}$.
    Azt mutatjuk meg, hogy ekkor $B$ oszloprendszere minimális generátorrendszere $A$ oszlop-vektorterének
    és $C$ sorrendszere minimális generátorrendszere $A$ sor-vektorterének.

    Vegyük észre, hogy $\srank{A}\leq \crank{A}$.
    Ugyanis ha az $A$ mátrix oszlop-vektorterének veszünk egy tetszőleges választott
    $b_1,\ldots,b_k$ generátorrendszerét, 
    akkor van $B\in\mathbb{F}^{n\times k}$ és $C\in\mathbb{F}^{k\times m}$ mátrix, hogy $A=BC$.
    Az $r=\srank{A}$ szám az ilyen $k$ számok legkisebbike, tehát valóban $r\leq\crank{A}$.
    \\
    Most tekintsük a (\dag)-ben rögzített szorzatot.
    Jelölje $W$ a $B$ mátrix és $V$ az $A$ mátrix oszlopvektorterét.
    Mivel $BC$ oszlopai $B$ oszlopainak lineáris kombinációja, 
    ezért $A$ minden oszlopa beleesik $W$-be, 
    így az $A$ oszlopainak lineáris burka is részhalmaza $W$-nek,
    azaz 
    \[
        V\subseteq W.
    \]
    A $B$ mátrixnak $r$ darab oszlopa van, 
    tehát $\dim W\leq r$.
    Látjuk tehát, hogy 
    \[\dim W\leq r\leq\crank{A}=\dim V,
    \]
    ami csak úgy lehetséges, 
    hogy $V=W$.
    A $B$ oszlopai tehát $V$-nek is generátorrendszerét alkotják,
    és $r$ minimalitása szerint egy elem sem elhagyható a generátorrendszer tulajdonság
    elvesztése nélkül.

    A sorokra vonatkozó indoklás analóg.
    Először is $\srank{A}\leq \rrank{A}$.
    Ugyanis ha az $A$ mátrix sorvektorterének veszünk egy tetszőleges választott
    $c_1,\ldots,c_k$ generátorrendszerét, 
    akkor van $B\in\mathbb{F}^{n\times k}$ és $C\in\mathbb{F}^{k\times m}$ mátrix, hogy $A=BC$.
    Az $r=\srank{A}$ szám az ilyen $k$ számok legkisebbike, tehát valóban $r\leq\rrank{A}$.
    \\
    Most tekintsük a (\dag)-ben rögzített szorzatot.
    Jelölje $W$ a $C$ mátrix és $V$ az $A$ mátrix sorvektorterét.
    Mivel $BC$ sorai $C$ sorainak lineáris kombinációja, 
    ezért $A$ minden sora $W$-be esik,
    így az $A$ sorainak lineáris burka is részhalmaza $W$-nek,
    azaz 
    \[
        V\subseteq W.
    \]
    A $C$ mátrixnak $r$ sora van, 
    tehát $\dim W\leq r$.
    Látjuk tehát, hogy 
    \[\dim W\leq r\leq\crank{A}=\dim V,
    \]
    ami csak úgy lehetséges, 
    hogy $V=W$.
    A $C$ oszlopai tehát $V$-nek is generátorrendszerét alkotják,
    és $r$ minimalitása szerint egy elem sem elhagyható a generátorrendszer tulajdonság
    elvesztése nélkül.

    Ezt kellett belátni. 
\end{proof}
\begin{definition}[mátrix rangja]\index{matrix@mátrix rangja}
    Mivel a sorrang, az oszloprang, a feszítőrang minden mátrix mellett azonos,
    ezért a továbbiakban a közös értékre a \emph{rang} szót is használjuk.\footnote{Lásd: \parencite{Wardlaw2005}}
\end{definition}
\begin{note}
    Érdemes a Rang-tétel következő összegzését megjegyezni.
    Legyen $A\in\mathbb{F}^{n\times m}$ mátrix, amelynek $r$ a rangja.
    Ekkor létezik $A=BC$ felbontása, ahol $B\in\mathbb{F}^{n\times r},C\in\mathbb{F}^{r\times m}$.
    Ez a felbontás persze nem egyértelmű, hiszen $A$ oszlop-vektorterének nagyon sok bázisa van.
    Viszont minden ilyen felbontásban $B$ oszloprendszere az $A$ oszlop-vektorterének, 
    míg $C$ sorrendszere az $A$ sorvektorterének minimális generátorrendszerét, ergo bázisát alkotja.
\end{note}
Következményképpen érdemes meggondolni a mátrix és inverzének felcserélhetőségére vezető állítást.
\begin{proposition}
    Legyenek $A,B\in\mathbb{F}^{n\times n}$ négyzetes mátrixok, amelyekre AB=I.
    Ekkor BA=I is teljesül.
\end{proposition}
\begin{proof}
    Az identitás mátrix rangja nyilván $n$.
    E mátrix feszítőrangjának definíciójára gondolva, 
    az előző megjegyzés szerint $A$ oszlopai $\mathbb{F}^n$ lineárisan független rendszerét alkotják.
    Vegyük észre, hogy a mátrix szorzás asszociativitását is kihasználva
    \[
        A\left( BA-I \right)=A\left( BA \right)-AI=\left( AB \right)A-AI=IA-AI=A-A=0.
    \]
    Na most, 
    ha $BA\neq I$ lenne, 
    akkor a $BA-I$ mátrixnak lenne egy olyan nem zérus oszlopa, 
    melynek elemeivel mint együtthatókkal képzett lineáris kombinációja az $A$ oszlopainak 
    a zéró vektort eredményezi.
    Ez persze ellentmond az $A$ oszloprendszer lineáris függetlenségének,
    tehát $BA=I$ valóban fennáll.%
    \footnote{%
        A feszítőrang fogalmának ismerete nélküli -- talán még elemibb -- bizonyítás: \parencite{doi:10.4169/college.math.j.48.5.366}%
    }%
\end{proof}
\begin{defprop}[invertálható mátrix]
    Legyen $A\in\mathbb{F}^{n\times n}$ egy négyzetes mátrix.
    Az alábbi feltételek egymással ekvivalensek.
    \begin{enumerate}
        \item Van egyetlen olyan $B\in\mathbb{F}^{n\times n}$ mátrix,
            amelyre $AB=I$,
        \item Van olyan $B\in\mathbb{F}^{n\times n}$ mátrix,
            amelyre $AB=I$,
        \item Van egyetlen olyan $B\in\mathbb{F}^{n\times n}$ mátrix,
            amelyre $BA=I$,
        \item Van olyan $B\in\mathbb{F}^{n\times n}$ mátrix,
            amelyre $BA=I$,
        \item $\rank A=n$,
        \item $A$ oszlopai lineárisan független rendszer alkotnak,
        \item $A$ sorai lineárisan független rendszert alkotnak.
    \end{enumerate}
    Ha a fenti feltételek egyike (ergo mindegyike) fennáll, 
    akkor azt mondjuk, hogy $A$ mátrix \emph{invertálható}\index{invertálható mátrix}.
    Szinonimaként használjuk még a \emph{nemszinguláris}\index{nemszinguláris mátrix}, 
    vagy az \emph{reguláris}\index{reguláris mátrix} szavakat.
    Ha egy mátrix nem invertálható, akkor \emph{szingulárisnak}\index{szinguláris mátrix} nevezzük.

    Egy invertálható négyzetes mátrix esetén azt az egyetlen $B$ mátrixot, amelyre
    \(
    AB=I
    \)
    fennáll az $A$ inverzének mondjuk, és $A^{-1}=B$-vel jelöljük.
    Világos, hogy
    \[
        AA^{-1}=I=A^{-1}A,\quad (A^{-1})^{-1}=A.\qedhere
    \]
\end{defprop}
\begin{proof}
    A 2., 4., 5., 6., 7. állítások ekvivalenciája nyilvánvaló az előzőek szerint.
    Ha $AB=I=AC$, akkor $A\left( B-C \right)=0$ így $A$ oszloprendszere lineáris függetlensége 
    miatt $B=C$. 
    Ezzel $2.\Rightarrow 1.$ implikációt is beláttuk.
    Az 1. és 3. feltevések ekvivalenciája az előző állítás miatt teljesül.
\end{proof}
\chapter*{Elemi fogalmak mégegyszer}
Tegyük fel, hogy, hogy ismerjük az alábbi fogalmakat:
\begin{enumerate}
    \item Vektorrendszer lineáris függetlensége;
    \item Altér, lineáris burok, generátorrendszer;
    \item Mátrix szorzás,
    \item Gauss\,--\,Jordan-elimináció.\index{Gauss\,--\,Jordan-elimináció}  Pontosan azt tesszük fel, hogy ha $Q$ egy négyzetes mátrix, melynek oszlopai
        lineárisan függetlenek, akkor az elemi sor műveletekkel az identikus mátrixszá transzformálható.
        Mivel az elemi sorműveletek, bal-szorzások alkalmas mátrixszal, 
        azt kapjuk, hogy létezik $P$ mátrix, amelyre $PQ=I$.
\end{enumerate}
Amit határozottan kerülünk, és azt tesszük fel, hogy nem ismerjük,
\begin{enumerate}
    \item Bázisok fogalma,
    \item Különböző bázisok azonos számossága,
    \item determináns.
\end{enumerate}
Külön probléma a mátrix rangjának definíciója.
Nem definiálhatom, 
mint az oszlop vagy sorvektortér dimenzióját, hiszen a felépítés jelen szintjén még nincs bázis.
Természetesen azt sem tudjuk még, hogy a maximális lineárisan független sor- vagy oszloprendszer választástól függetlenül mindig azonos elemszámú.
A mátrix feszítőrangja viszont definiálható.
\begin{definition}[mátrix feszítőrangja]
    Legyen $A\in\mathbb{F}^{n\times m}$ egy tetszőleges nem zérus mátrix.
    Azt mondjuk, hogy feszítőrangja $r$, ha $r$ a legkisebb olyan pozitív egész, amelyre $A$ előáll
    \[
        A=BC
    \]
    alakban, ahol $B\in\mathbb{F}^{n\times r}$ és $C\in\mathbb{F}^{r\times n}$.
\end{definition}
Világos, hogy tetszőleges nemzérus négyzetes mátrixra ez jól definiált és $1\leq r \leq \min\{n,m\}$.
A Rang-tételnek a szokásosnál egy kicsit erősebb formáját lehet megfogalmazni a dimenzió fogalmának bevezetése nélkül,
ami a lenti 3. állítás.

\begin{proposition}
    Az alábbi állítások egymás következményei:
    \begin{enumerate}
        \item Homogén lineáris egyenletrendszernek, amelynek több ismeretlene van, mint egyenlete
            mindig létezik nem triviális megoldása.
        \item Lineárisan független vektorrendszer elemszáma nem nagyobb mint egy generátorrendszer elemszáma.
        \item Minden nemzérus mátrixban 
            a maximális lineárisan független oszloprendszerek 
            és maximális lineárisan független sorrendszerek azonos elemszámúak, 
            és ez a szám egybeesik a mátrix feszítőrangjával.
        \item
            Legyen $A,B\in\mathbb{F}^{n\times n}$ négyzetes mátrixok.
            Ekkor $AB=I$ esetén $BA=I$ is fennáll.\qedhere
    \end{enumerate}
\end{proposition}
\begin{proof}[1. \Rightarrow 2.]
    Legyen $\left\{ y_1,\ldots,y_n \right\}$ egy generátorrendszer,
    és $\left\{ x_1,\ldots,x_m \right\}$ olyan vektorrendszer a vektortérben, ahol $m>n$.
    Meg kell mutatnunk, hogy ez utóbbi egy lineárisan összefüggő.
    Világos, hogy minden $1\leq k\leq m$ mellett
    \[
        x_k=\sum_{j=1}^n\alpha_{j,k}y_j.
    \]
    Egyelőre tetszőleges $\xi_1,\ldots,\xi_m$ együtthatók mellett
    \begin{eqnarray}
        \sum_{k=1}^m\xi_kx_k=
        \sum_{k=1}^m\sum_{j=1}^n\xi_k\alpha_{j,k}y_j=
        \sum_{j=1}^n\left( \sum_{k=1}^m\alpha_{j,k}\xi_k \right)y_j
        \label{eq:sys}
    \end{eqnarray}
    Tekintsük az $\left( \alpha_{j,k} \right)$ együtthatók generálta
    homogén lineáris egyenletrendszert. 
    Itt $j=1,\ldots,n$ és $k=1,\ldots,m$.
    Mivel $m>n$, ezért az ismeretlenek száma több mint az egyenletek száma.
    Létezik tehát nem triviális megoldás\index{triviális megoldás}, azaz léteznek nem mind nulla
    $\xi_1,\ldots,\xi_m$ számok, amelyekre minden $j=1,\ldots,n$ esetén
    \[
        \sum_{k=1}^m\alpha_{j,k}\xi_k=0.
    \]
    Találtunk tehát az $\left\{ x_1,\ldots,x_m \right\}$ vektorrendszernek egy
    nem triviális, 
    de a zéró vektort eredményező,
    lineáris kombinációját (\ref{eq:sys}).
\end{proof}
\begin{proof}[2.\Rightarrow 3.]
    Jelölje $r$ az $A\in\mathbb{F}^{n\times m}$ mátrix feszítőrangját.
    Legyen $r_c$ a mátrix egyik rögzített maximális lineárisan független oszloprendszerének elemszáma.
    \begin{itemize}
        \item 
            Ezen oszlopokat egy $B\in\mathbb{F}^{n\times r_c}$ mátrixba téve 
            -- a maximalitás miatt -- létezik olyan $C\in\mathbb{F}^{r_c\times m}$ mátrix, 
            amelyre $A=BC$, azaz $r\leq r_c$.
        \item
            Most tekintsünk egy tetszőleges olyan
            \(
            A=BC
            \)
            felbontást, 
            ahol $B\in\mathbb{F}^{n\times r}$ és $C\in\mathbb{F}^{r\times m}$.
            Jelölje $W$ a $B$ mátrix oszlopai lineáris burkát. 
            Az $A$ mátrix fent rögzített maximális lineárisan független oszloprendszere egy lineárisan független rendszer a 
            $W$ vektortérben,
            és $B$ oszlopai pedig egy generátorrendszer ugyanebben a vektortérben.
            A 2. állitás szerint $r_c\leq r$.
    \end{itemize}
    Evvel megmutattuk, hogy bármely két maximális lineárisan független oszloprendszer azonos elemszámú, és számuk megegyezik a mátrix feszítőrangjával.


    Legyen $r_w$ az $A$ mátrix egyik rögzített maximális lineárisan független sorrendszerének elemszáma.
    \begin{itemize}
        \item 
            Ezen sorokat egy $C\in\mathbb{F}^{r_w\times m}$ mátrixba téve 
            -- a maximalitás miatt -- létezik olyan $B\in\mathbb{F}^{n\times r_w}$ mátrix, 
            amelyre $A=BC$, azaz $r\leq r_w$.
        \item
            Most tekintsünk egy tetszőleges olyan
            \(
            A=BC
            \)
            felbontást, 
            ahol $B\in\mathbb{F}^{n\times r}$ és $C\in\mathbb{F}^{r\times m}$.
            Jelölje most $V$ a $C$ mátrix sorai lineáris burkát. 
            Az $A$ mátrix fent rögzített maximális lineárisan független sorrendszere egy lineárisan független rendszer e
            $V$ vektortérben,
            és $C$ sorai pedig egy generátorrendszert alkotnak ugyanebben a $V$ vektortérben.
            A 2. állitás szerint $r_w\leq r$.
    \end{itemize}
    Evvel azt is megmutattuk, 
    hogy bármely két maximális lineárisan független sorrendszer azonos elemszámú, 
    és számuk megegyezik a mátrix feszítőrangjával.
\end{proof}
\begin{proof}[3.\Rightarrow 4.]
    Tegyük fel, hogy $AB=I$.
    Az identitás mátrix rangja nyilván $n$.
    A 3. állitás miatt a feszítőrang is $n$.
    Ha $A$ oszlopai nem lennének lineárisan függetlenek,
    akkor lenne $A=CD$ felbontás, ahol $C\in\mathbb{F}^{n \times r}$ és $D\in\mathbb{F}^{r\times n}$ valamely $r<n$ mellett.
    Ekkor persze $I=AB=\left( CD \right)B=C\left( DB \right)$ is teljesülne, 
    ahol $DB\in\mathbb{F}^{r\times n}$ ellentmondva az identitás mátrix feszítőrangja definíciójának.
    Világos, hogy
    \[
        A\left( BA-I \right)=
        A\left( BA \right)-AI=
        \left( AB \right)A-AI=IA-AI=0.
    \]
    Figyelembe véve, hogy $A$ oszlopai lineáris függetlenek, ez csak úgy lehetséges, ha $BA-I$ a zéró mátrix, ergo $BA=I$.
\end{proof}
\begin{proof}[4.\Rightarrow 1.]
    Legyen $A\in\mathbb{F}^{n\times m}$ a homogén lineáris egyenletrendszer együttható mátrixa,
    ahol $n$ az egyenletek száma, $m$ az ismeretlenek száma.
    Azt kell megmutatnunk, hogy az oszloprendszer lineárisan összefüggő.
    Ha független lenne, akkor
    egészítsük ki e mátrixot alulról $m-n$ darab csupa nullákat tartalmazó sorral.
    Mivel $m>n$ ezért a kiegészített $Q\in\mathbb{F}^{m\times m}$ mátrix legalsó sora csak nullát tartalmaz.
    Mivel $A$ oszlopai lineárisan függetlenek, ezért $Q$ oszlopai is azok.
    Emiatt létezik $P\in\mathbb{F}^{m\times m}$ mátrix, amelyre $PQ=I$.
    A 3. állitás szerint $I=QP$ is teljesül, 
    ami azt jelenti, hogy $I$ utolsó sora a csupa nullákat tartalmaz, ami ellentmondás.
    Beláttuk tehát, hogy $A$ oszlopai lineárisan összefüggőek, azaz az eredeti egyenletrendszernek van nem triviális megoldása.
\end{proof}
\end{document}
